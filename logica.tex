\documentclass[12pt,a4paper]{article}

\usepackage[italian]{babel}
\usepackage[T1]{fontenc}
\usepackage[utf8]{inputenc}

\usepackage[left=2cm, right=2cm, top=2cm, bottom=2cm]{geometry}
\usepackage{caption}
\usepackage{amsmath}
\usepackage{titlesec}

\titlespacing\subsection{0pt}{70pt}{5pt}
\newcommand{\Setsuchthat}{\;\ifnum\currentgrouptype=16 \middle\fi|\;} % https://github.com/gjkf/Algebra-1/blob/master/plot-snippets.tex

\title{Matematica discreta}
\author{Stefano Mecocci}
\date{2018/2019}

\setcounter{tocdepth}{3}

\begin{document}
  \maketitle
  \thispagestyle{empty}

  \newpage
  \tableofcontents
  \newpage

  \section{Insiemi}
  L'insieme è la base su cui il resto delle strutture matemmatiche sono definite. Esso è definito come una collezione di oggetti.

  \subsection{Rappresentazioni}
  Il modo più elementare per rappresentare gli insiemi consiste nell'usare il diagramma di Venn, ma i metodi più usati sono: di elencazione e di proprietà.

  \begin{equation}
    A = \{1, 2, 3, ...\}
  \end{equation}
  \captionof{figure}{Rappresentazione per elencazione}

  \begin{equation}
    A = \{x \in X \Setsuchthat P(x)\}
  \end{equation}
  \captionof{figure}{Rappresentazione per proprietà}

  \subsection{Basi}
  Negli appunti sono presenti alcuni simboli speciali di seguito spiegati:

  \begin{itemize}
    \item $ \emptyset $ è l'insieme vuoto equivalente a $ \{\} $
    \item $ \forall x $ significa "per ogni x"
    \item $ \exists x $ significa "esiste almeno un x"
    \item $ \Setsuchthat $ significa "tale che"
    \item $ x \in X $ significa "x appartiene ad X"
    \item quando si "sbarra" un simbolo in generale si intende l'opposto(es. $ \neq $ significa non uguale)
  \end{itemize}

  \textbf{Attenzione:} Bisogna stare attenti ad alcune denotazioni, ovvero tenendo conto che $ A = \{a, b, c\} $:

  \begin{itemize}
    \item $ \emptyset \neq \{\emptyset\} $
    \item $ a \in A $ è vero
    \item $ \{a\} \subseteq A $ è vero
    \item $ \{a\} \in A $ è \textbf{falso}
  \end{itemize}

  Inoltre sono presenti anche riferimenti ad insiemi conosciuti:

  \begin{itemize}
    \item $ N = \{0, 1, 2, 3, ...\} $ è l'insieme dei numeri naturali
    \item $ Z = \{ ... -2, -1, 0, 1, 2, 3, ...\} $ è l'insieme dei numeri interi
    \item $ Q  = \{\frac{a}{b} \Setsuchthat a,b \in Z\:e\:b \neq 0\} $ é l'insieme dei numeri razionali
    \item $ R $ é l'insieme dei numeri reali
  \end{itemize}
  
  Alcuni esempi per chiare i simboli, osservando la definizione dell'insieme B

  \begin{equation}
    B = \{x \in Z \Setsuchthat x\ap{2} < 1\}
  \end{equation}

  Si può affermare che:
  \begin{itemize}
    \item $ 1 \notin B $ è vera
    \item $ \forall x \in B \Setsuchthat x > 10 $ é falsa 
  \end{itemize}

  % sottosezione
  \subsection{Sottoinsiemi}
  Prendendo come riferimento l'insieme

  \begin{equation}
    A = \{4, 6, 8, 10, 12, ...\}
  \end{equation}

  si può affermare che $ A \subseteq N $ é vera in quanto dice che "A è un sottoinsieme di N". Se andiamo a controllare gli elementi presenti in A li ritroviamo in N. Potremmo anche dire che:

  \begin{itemize}
    \item $ \forall n \in A \Setsuchthat \text{n é un multiplo di 3} $ è falsa, infatti non è vero che \textbf{tutti} gli elementi di A sono multipli di 3

    \item $ \exists n \in A \Setsuchthat \text{n é un multiplo di 3} $ è vera, infatti è vero che \textbf{esiste almeno un elemento} di A che sia multiplo di 3
  \end{itemize}
  
\end{document}