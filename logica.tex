\documentclass[12pt,a4paper]{article}

\usepackage[italian]{babel}
\usepackage[T1]{fontenc}
\usepackage[utf8]{inputenc}

\usepackage[left=2cm, right=2cm, top=2cm, bottom=2cm]{geometry}
\usepackage{caption}
\usepackage{amsmath,amsfonts}
\usepackage{titlesec}

\titlespacing\subsection{0pt}{70pt}{5pt}
\newcommand{\Setsuchthat}{\;\ifnum\currentgrouptype=16 \middle\fi|\;} % https://github.com/gjkf/Algebra-1/blob/master/plot-snippets.tex

\title{Matematica discreta}
\author{Stefano Mecocci}
\date{2018/2019}

\setcounter{tocdepth}{3}

\begin{document}
  \maketitle
  \thispagestyle{empty}

  \newpage
  \tableofcontents
  \newpage

  \section{Insiemi}
  L'insieme è la base su cui il resto delle strutture matemmatiche sono definite. Esso è definito come una collezione di oggetti.

  \subsection{Rappresentazioni}
  Il modo più elementare per rappresentare gli insiemi consiste nell'usare il diagramma di Venn, ma i metodi più usati sono: di elencazione e di proprietà.

  \begin{equation}
    A = \{1, 2, 3, \ldots\}
  \end{equation}
  \captionof{figure}{Rappresentazione per elencazione}

  \begin{equation}
    A = \{x \in X \Setsuchthat P(x)\}
  \end{equation}
  \captionof{figure}{Rappresentazione per proprietà}

  \subsection{Basi}
  Negli appunti sono presenti alcuni simboli speciali di seguito spiegati:
  \begin{center}
    \begin{tabular}{p{2cm}p{10cm}}
      $ \emptyset $ & è l'insieme vuoto equivalente a $ \{\} $\\
      $ \forall x $ & significa ``per ogni $x$''\\
      $ \exists x $ & significa ``esiste almeno un $x$''\\
      $ \Setsuchthat $ & significa ``tale che''\\
      $ x \in X $ & significa ``$x$ appartiene ad $X$''\\
    \end{tabular}
  \end{center}
  Quando si ``sbarra'' un simbolo in generale si intende l'opposto (es. $ \neq $ significa
  non uguale)\\[\baselineskip]
  \textbf{Attenzione:} Bisogna stare attenti ad alcune denotazioni, ovvero tenendo conto che $ A = \{a, b, c\} $:

  \begin{itemize}
    \item $ \emptyset \neq \{\emptyset\} $
    \item $ a \in A $ è vero
    \item $ \{a\} \subseteq A $ è vero
    \item $ \{a\} \in A $ è \textbf{falso}
  \end{itemize}

  Inoltre sono presenti anche riferimenti ad insiemi conosciuti:

  \begin{itemize}
    \item $ \mathbb{N} = \{0, 1, 2, 3, \ldots\} $ è l'insieme dei numeri naturali
    \item $ \mathbb{Z} = \{ \ldots -2, -1, 0, 1, 2, 3, \ldots\} $ è l'insieme dei numeri interi
      \item $ \mathbb{Q}  = \{\frac{a}{b} \Setsuchthat a,b \in \mathbb{Z}\:\text{e}\:b \neq 0\} $ è l'insieme dei numeri razionali
      \item $ \mathbb{R} $ è l'insieme dei numeri reali
  \end{itemize}

  Alcuni esempi per chiare i simboli, osservando la definizione dell'insieme B

  \begin{equation}
    B = \{x \in \mathbb{Z} \Setsuchthat x^{2} < 1\}
  \end{equation}

  Si può affermare che:
  \begin{itemize}
    \item $ 1 \notin B $ è vera
    \item $ \forall x \in B \Setsuchthat x > 10 $ è falsa
  \end{itemize}

  % sottosezione
  \subsection{Sottoinsiemi}
  Prendendo come riferimento l'insieme

  \begin{equation}
    A = \{4, 6, 8, 10, 12, \ldots\}
  \end{equation}

  si può affermare che $ A \subseteq \mathbb{N} $ è vera in quanto dice che ``$A$ è un sottoinsieme
  di $\mathbb{N}$''. Se andiamo a controllare gli elementi presenti in $A$ li ritroviamo
  in $\mathbb{N}$. Potremmo anche dire che:

  \begin{itemize}
    \item $ \forall n \in A \Setsuchthat \text{n è un multiplo di 3} $ è falsa, infatti non è vero che \textbf{tutti} gli elementi di A sono multipli di 3

    \item $ \exists n \in A \Setsuchthat \text{n è un multiplo di 3} $ è vera, infatti è vero che \textbf{esiste almeno un elemento} di A che sia multiplo di 3
  \end{itemize}
\end{document}
